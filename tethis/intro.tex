\chapter*{Введение}

В настоящее время происходит повсеместное развитие ДНК-технологий с целью применения
в различных сферах человеческой деятельности, что может быть обусловлено более высокой
эффективностью по сравнению с ранее применяемыми технологиями и методами.
Применение технологий ДНК-идентификации в области криминалистики с целью определения,
например, этногеорафической принадлежности индивида потенциально может облегчить задачу сужения зоны поиска
преступников, концентрируя усилия правоохранительных органов и обеспечивая оперативность раскрытия преступлений
при одновременном снижении материальных затрат.

Основное внимание посвящено созданию в Союзном государстве технологии ДНК-идентификации и
определения этногеографической принадлежности неизвестного индивида по его ДНК,
в том числе полученной из биологических следов, и создание базы данных по коплексу генетических маркеров,
дифференцирующих генофонды популяций населения Союзного государства для получения информации, которая может быть
использована в криминалистике.

В данной магистерской диссертации будут рассмотрены вопросы проектирования и реализации
програмного обеспечения для ДНК-идентификации. В частности, в главе 1 будут рассмотрены основные понятия
и общая схема ДНК-идентификации. В главе 2 будет описан разработанный метод получения ДНК-маркеров (аннотации)
для прошедших стадию секвенирования биологических материалов. В главе 3 рассматривается проблема определения
этногеографической принадлежности индивидов на основе ДНК-маркеров на территории Республии Беларусь.
В главе 4 содержится описание разработанного метода по определению географической принадлежности
индивидов на основе ДНК-маркеров на территории Республии Беларусь.

Комплекс разработанных методов и програмного обеспечения может быть взят за основу для создания
технологий ДНК-идентификации для определения географической принадлежности неизвестных индивидов.
Програмная реализация всех разработанных методов выполнена с использованием языка Python
и размещена в публичном репозитории на github. Проведены вычислительные эксперименты с целью
проверки и демонстрации работы реализованных методов, их сравнения.
