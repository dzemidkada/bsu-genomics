\chapter*{ОБЩАЯ ХАРАКТЕРИСТИКА РАБОТЫ}
Магистерская диссертация с.,  рис.,  таблиц, 20 источников.

Ключевые слова: ДНК, ИДЕНТИФИКАЦИЯ, ЛОКУС, АЛЛЕЛЬ, STR-МАРКЕРЫ, ЭТНОГЕОГРАФИЧЕСКАЯ ПРИНАДЛЕЖНОСТЬ, КЛАССИФИКАЦИЯ.

Объект исследования – методы выделения маркеров из последовательностей нуклеотидов, ДНК-идентификация.

Цель работы – изучение методов работы с STR-маркеров, применение методов машинного обучения
для классификации генотипов.

Методы исследования – анализ, эксперимент, тестирование, сравнение.

Результаты исследования:
\begin{itemize}
\item изучены методы работы с последовательностями нуклеотидов
\item разработаны методы поиска шаблонов STR-маркеров, их аннотации, реализовано веб-приложение предоставляющее доступ к разработанным инструментам
\item разработаны методы определения географической принадлежности, разработано вспомогательное веб-приложение
\item разработано вспомогательное веб-приложение для исправления разметки в анкетных данных
\item с помощью методов машинного обучения произведена оценка качества определения этногеографического происхождения
\item реализованы базовые компоненты для ДНК-идентификации
\end{itemize}

Область применения – решение задач ДНК-идентификации.

\chapter*{АГУЛЬНАЯ ХАРАКТАРЫСТЫКА РАБОТЫ}

Магiстарская дысертацыя  ст.,  мал.,  таблiц, 20 крынiц.

Ключавыя словы: ДНК, ДЭНТЫФІКАЦЫЯ, ЛОКУСЫ, АЛЕЛЬ, STR-МАРКЕР, ЭТНОГЕОГРАФИЧЕСКАЯ ПРЫНАЛЕЖНАСЬЦЬ, КЛАСІФІКАЦЫЯ.

Аб'ект даследавання - метады вылучэння маркераў з паслядоўнасцяў нуклеатыдаў, ДНК-ідэнтыфікацыя.

Мэта работы - вывучэнне метадаў работы з STR-маркераў, прымяненне метадаў машыннага навучання
для класіфікацыі генатыпаў.

Метады даследавання - аналіз, эксперымент, тэставанне, параўнанне.

Вынікі даследавання:
\begin{itemize}
\item вывучаны метады працы з паслядоўнасцямі нуклеатыдаў
\item распрацаваны метады пошуку шаблонаў STR-маркераў, іх анатацыі, рэалізавана вэб-дадатак якое прадастаўляе доступ да распрацаваных інструментаў
\item распрацаваны метады вызначэння геаграфічнай прыналежнасці, распрацавана вспомоготельное вэб-дадатак
\item распрацавана вспомоготельное вэб-дадатак для выпраўлення разметкі ў анкетных дадзеных
\item з дапамогай метадаў машыннага навучання праведзеная ацэнка якасці вызначэння этногеографического паходжання
\item рэалізаваны базавыя кампаненты для ДНК-ідэнтыфікацыі
\end{itemize}


\chapter*{GENERAL CHARACTERISRICS OF THE WORK}

Master thesis p., pic., tables, 20 resources.

Key words: DNA, IDENTIFICATION, LOCUS, ALLELE, STR MARKERS, ETHNO-GEOGRAPHIC ORIGIN, CLASSIFICATION.

Object of study - methods of markets retrieval from nucleotide sequences, DNA identification.

The purpose of this work is to study methods of working with STR markers, the use of machine learning methods
for the classification of genotypes.

Research methods - analysis, experiment, testing, comparing.

The results of the study:
\begin{itemize}
\item nucleotide sequence processing methods were studied
\item methods of STR markers patterns retrieval, annotation approach, a web application that provides access to developed tools was implemented
\item methods of determining geographic location, auxiliary web application was developed
\item an auxiliary web application for correcting markups in annotated data was developed
\item using machine learning methods assessed the quality of determining ethno-geographical origin
\item implemented basic components for DNA identification
\end{itemize}

Field of application - tasks of DNA identification.
