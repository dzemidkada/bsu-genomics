\chapter{Определение географической принадлежности}

Рассматривая практические задачи, возникающие в контексте ДНК-идентификации, можно
выделить задачу определения географической принадлежности. Основным отличием данной задачи
от задачи определения этногеографичиской принадлежности является
большая гранулярность ожидаемых результатов, например, координаты населенного пункта проживания (места рождения)
неизвесного индивида. В данном случае множество координат-кандидатов не ограничено конечным множеством вариантов.

\section{Постановка задачи и метрики качества}
Задача определения географической принадлежности в общем виде может иметь следующий вид:
для заданного генотипа неизвесного индивида необходимо определить наиболее вероятные
места его происхождения. Очевидно, что данное определение носит весьма общий характер,
что в свою очередь заметно усложняет введение адекватных метрик качества.

Для создания более четкой форлулировки определение выше можно параметризовать, например,
с помощью следующих параметров:
\begin{itemize}
\item Погрешность (далее $d$) - некоторая величина, расстояние, характеризующая актуальность
предоставленных вариантов. Так, например, методы с погрешностью более 1000 км вряд ли
будут иметь спрос на территории Республики Беларусь. Мотивация для введения данного параметра проста:
чем точнее будут предоставленные варианты, тем лучше.

\item Максимальное количество кандидатов (далее $k$). Интуитивно понятно, что метод
определения географической принадлежности будет полезен на практике только при условии
предоставленния ограниченного числа наиболее вероятных координат-кандидатов. Иначе можно было бы
просто покрыть равномерной решеткой всю целевую область - истинные координаты были бы
в достаточной близости от одного из узлов.
\end{itemize}

Таким образом, можно немного сократить круг исследуемых задач:
нас интересуют методы, способные для некоторого генотипа $g$ предоставить список из $k$
наиболее вероятных координат-кандидатов согласно некоторой метрике качества $Q_{d}_{k} \left( C, c_{g} \right)$,
где $C = \left(c_{1}, ..., c_{k} \right)$ - множество кандидатов, $c_{g}$ - истинные координаты.

Далее представлены некоторые варианты метрик качества:
\begin{itemize}
\item $Q^{hit}_{d}_{k} \left( C, c_{g} \right) = \max_{c \in C} \mathbbm{1} \left[ ||c - c_{g}|| <= d \right]$

Индикатор попадания - хотя бы один из кадидатов находится в достаточной близости от истинных координат генотипа.

\item $Q^{precision}_{d}_{k} \left( C, c_{g} \right) = \frac{1}{k} \sum_{c \in C} \mathbbm{1} \left[ ||c - c_{g}|| <= d \right]$

Точность - доля достаточно близких к истинным координатам генома кандидатов.

\item $Q^{mean}_{d}_{k} \left( C, c_{g} \right) = \frac{1}{k} \sum_{c \in C} ||c - c_{g}||$

Среднее растояние от кандидатов до истинных координат.

\item $Q^{median}_{d}_{k} \left( C, c_{g} \right) = Median_{c \in C} ||c - c_{g}||$

Медиана расстояний от кандидатов до истинных координат.
\end{itemize}

Метрики выше подходят для оценки качества предсказания кандидатов для одного генотипа.
На практике возникает необходимость оценивать качество работы методов на множестве (например, тестовом)
генотипов. Довольно полезными оказываются следующие метрики:
\begin{itemize}
\item Покрытие - мощность множества различных кандидатов, которые были предсказаны
для некоторого множества различных генотипов. Методы предсказания географической принадлежности должны обладать
достаточной вариабильностью. Если множества кандидатов для различных генотипов будут
обладать достаточно большим множеством общих кандидатов, то скорее всего роль генотипа
в генерации кандидатов играет недостаточно серьезную роль.
\item Компактность - желательно, чтобы все кандидаты располагались в достаточной близости друг от друга.
Ясно, что данное требование может противоречить требованию наличия покрытия достаточной величины.
\end{itemize}

\section{Методы решения и тестирования}

Для разработки и тестирования методов решения задачи описанной в предыдущем разделе,
общая база генотипов была разделена на три части:

\begin{itemize}
\item Тренировочная выборка, далее $X_{train}$, содержит 1169 аннотированных генотипов
\item Валидационная выборка, далее $X_{val}$, содержит 62 аннотированных генотипов
\item Тестовая выборка, далее $X_{test}$, содержит 65 аннотированных генотипов
\end{itemize}

Каждый образец будем представлять в виде $(g_i, c^i_{g})$ - генотип и координаты населенного пункта,
где родился соответствующий индивид. Стоит отметить, что при разбиении общей базы генотипов
на выборки, скорее всего эффект на разбиение оказал влияние тот факт,
что из-за частоты встречаемости в исходной базе генотипов, с большей вероятностью
в валидационную и тестовую выборки могли попасть генотипы индивидов густонаселенных районов.

Таким образом предполагаем, что валидационная выборка будет использоваться для подбора гиперпараметров,
либо отбора моделей. А финальное тестирование будет производится на тестовой выборке.

Предлагается рассмотреть модели для определения географической пренадлежности генотипов, удовлетворяющих следующим свойствам:
\begin{itemize}
\item Модель состоит из 2 компонент: генератора кандидатов и модуля для ранжирования кандидатов.
Декомпозиция такого рода позволит упростить проектирование и разработку каждой из компонент
\item Генератор кандидатов, используя тренировочную выборку $X_{train}$, создает множество потенциальных
координат-кандидатов
\item Множество кандидатов ранжируется, используя тренировочную выборку $X_{train}$ и генотип-запрос,
$k$ кандидатов с максимальным рангом используются в качестве предсказаний модели
\end{itemize}

Существует 2 довольно простых способа генерации координат-кандидатов. Первый способ
заключается в извлечении из тренировочной выборки координат всех известных генотипов,
их дедублицирования. Второй способ - простая генерация узлов решетки, равномерно покрывающей
интересующую территорию (в данном случае Республики Беларусь). Второй способ подразумевает
наличие параметра отвечающего за количество узлов решетки, что дает проектируемому методу
дополнительную степень свободы.

Рассмотрим следующие базовые методы ранжирования координат-кандидатов:
\begin{itemize}
\item Случайный выбор $k$ кандидатов. Данный метод можно использовать для анализа значений метрик,
описанных в предыдущем разделе. Хороший метод ранжирования должен давать лучшие значения по
метрикам в сравнении со случайным выбором. Также, метрики данного метода ранжирования можно считать
пороговыми и ориентироваться на них.

\item Среди генотипов тренировочной выборки, выполняется поиск наиболее близкого к генотипу запроса $g_{query}$:

$ j_{min} = arg\,min_{\left(g^j_{train}, c^j_{train}\right) \in X_{train}} ||g^j_{train} - g_{query}||$,

наибольший ранг присваивается наиболее близким $c^{j_{min}}_{train}$ к кандидатам.

\item Среди генотипов тренировочной выборки выбираем $k$ наиболее близких к генотипу запроса $g_{query}$.
\end{itemize}

\section{Практические результаты}

В качестве практических экспериментов была проведена оценка качества следующих комбинаций:
\begin{itemize}
\item Генератор: равномерная сетка размером 20x20 узлов. Метод ранжирования: случайный выбор.

\item Генератор: координаты генотипов из тренировочной выборки. Метод ранжирования:
географическая близость к лучшему совпадению по генотипу.

\item Генератор: координаты генотипов из тренировочной выборки. Метод ранжирования:
ближайшие генотипы.
\end{itemize}
\section{Вывод}
