\chapter{Общая схема ДНК-идентификации}

%TODO
\section{Основные понятия}

%[Геном]

%[ДНК]

%[NGS sequencing]

%[типы хромосом]

Геномом называют сегмент ДНК, который участвует в кодировании белков, определяющих
некоторую особенность (например, цвет глаз). Локусом называют специальную область в хромосоме,
которая содержит ген некоторой особенности. В диплоидных организмах хромосомы всегда
разбиты по парам. Так, человек имея 46 хромосом, имеет 23 пары хромосом.
Каждая пара называется гомологической, так как одна из хромосом принадлежала материнской гамете, а другая отцовской.
Гомологические хромосомы имеют следующие свойства:
\begin{itemize}
\item Одинаковый размер и структура.
\item Содержат в себе гены, которые кодируют одни и те же свойства. Такие гены называют аллелями.
\end{itemize}
Организм называют гетерозиготным относительно некоторого свойства, если один из аллелей является доминантным,
тогда как другой - рецессивным.
Организм называют гомозиготным относительно некоторого свойства, если оба аллеля является доминантными,
либо рецессивными.
Генотипом организма называют множество содержащихся в нем генов.
Фенотипом называют множество физических свойств организма, обусловленных генотипом.

%[типы маркеров, свойства]
%Микросателлиты

%non-coding regions

%Работают с короткими стр повторами, т.к. там меньше шанс получить мутацию

\section{Общая схема ДНК-идентификации}

Рассмотрим базовые вопросы, которые могут возникнуть при построении технологии ДНК-идентификации:
\begin{itemize}

\item \textit{Получение, хранение биоматериалов}

Несложно представить себе типичные источники биоматериалов: в задачах криминалистики, например,
может осуществляется анализ места преступления, найденные при этом образцы необходимо сохранить
для дальнейшего использования. Данный случай можно классифицировать как реактивный, с точки зрения
целей и мотивов получения биоматериалов. С другой стороны, уже довольно хорошо заметна актуальность инициатив
по созданию баз данных, которые смогут помочь исследователям лучше понимать структуры генофондов,
а также находить практические применения и приносить пользу в широком спектре задач.
Хранение собранных биоматериалов осуществляется в специализированных лабораториях,
способных обеспечить максимальную сохранность биоматериалов и поддержание оптимальных услових для
их хранения.

\item \textit{Секвенирование генома, отдельных хромосом, специальных участков}

Вопрос получения цифрового представления генома человека требует внимательного изучения и анализа
требований. Учитывая стоимость, например, полногеномного секвенирования, а также характеристики результирующей
информации - около 3 миллиардов пар нуклеотитов, с учетом 'мусорной ДНК',
которая не принимает активного участия в кодировании белков.
Оправданным шагом может случить секвенирование лишь некоторой малой части генома,
содержащие все необходимые для исследований локусы. При таком таргетированном секвенировании, выбор
множества локусов может основываться на общепринятых стандартах в контексте решаемых задач,
например, набор локусов CODIS core для задач криминалистики.

Способы хранения результатов секвенирования могут зависить от конкретных задач и областей применения.
Так, например, в случае извлечения из прочитанных последовательностей некоторого набора маркеров,
помещение всего набора прочтений в базу (реляционную, например) для некоторого локуса
может не быть оправданным, так как процесс аннотации для этого множества данных будет произведен
всего один раз, таким образом естественно может отпасть необходимость хранения данных
в непосредственной близости к пользователям и предоставления им многократного доступа в целях обработки данных.
В данном случае прочтения можно отнести к категории так называемых архивных (англ. archived) данных.
Если же требования не ограничиваются работой только с высокоуровневыми признаками,
то схема хранения и доступа к данным может существенно измениться.

\item \textit{Аннотация секвенированных последовательностей на основе референсной информации}

В задачах обработки геномной информации, а именно прочтений, зачастую возникает необходимость в
сокращении объема исследуемых данных. В контексте работы с STR-маркерами возникает задача перехода
от множества прочтений последовательности нуклеотидов для некоторого локуса непосредственно к аллелям -
конкретным значениям, которые характеризуют найденную мощность паттерна.
Стоит отметить, что для успешной аннатации STR-маркеров необходимо иметь искомый паттерн. Данная информация
может быть получена из научных публикаций, либо открытых баз данных.
Аннотация SNP-маркеров имеет более простой вид - необходимо понять, совпадает ли нуклиотид с референсным.

\item \textit{Построение аналитических систем}

В настоящее время существует большой спектр задач по анализу геномных данных.
Например, установление идентичности двух биологических образцов (по средствам выделения и анализ маркера),
поиск наиболее схожих образцов, установление популяционной принадлежности.
При проектировании аналитических систем (систем хранения, систем интерактивной аналитической обработки)
необходимо учитывать специфику не только задачи, но и данных, на основе которых будет проводится анализ.
Так, например, платформа для анализа SNP-маркеров может существенно отличатся от платформы для анализа STR-маркеров
как функциональными требованиями, так и нефункциональными, ввиду того что количество рассматриваемых SNP-маркеров
может в десятки/тысячи раз превышать аналогичное число STR-маркеров.

\end{itemize}

В рамках данной магистерской диссертации будут рассмотрены следующие задачи,
являющиеся составными компонентами технологии ДНК-идентификации:

\begin{itemize}

\item Работа с секвенированными прочтениями.
Извлечение необходимых участков генома (локусов). Аннотация STR-маркеров.

\item Анализ STR-маркеров c целью установления возможности этногеографической идентификации.

\item Анализ STR-маркеров c целью определения географической принадлежности.

\end{itemize}

\section{Выводы}
