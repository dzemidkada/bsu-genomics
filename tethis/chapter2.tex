\chapter{Аннотация STR-маркеров}

Данная глава посвящена решению задачи поиска микросателлитов в последовательностях нуклеотидов.

В общем случае имеем следующую постановку задачи:
для некоторого образца $S$, по имеющимся результатам секвенирования его генома (возможно частичного),
необходимо произвести аннотацию STR-маркера в некотором локусе $L$.

\begin{Example}
Аннотация локуса D5S2800 может иметь представленный в таблице $\left(\ref{table:loc_ann_ex}\right)$ вид.
\end{Example}

На практике данную задачу можно свести к последовательности подзадач:
\begin{itemize}
\item Предобработка исходных данных. Из последовательности прочтений полученных в результате секвенирования
необходимо выделить только пересекающиеся с заданным локусом.
\item Зачастую в результате анализа литературы и научных публикаций удается выяснить,
что собой представляет паттерн STR-маркера для заданного локуса. Но иногда требуется
определить наиболее подходящие паттерны исходя из имеющихся прочтений.
\item По имеющимся прочтениям заданного локуса и определенному паттерну требуется
выяснить численные значения аллелей, соответствующие паттерну, а также строковые - аннотации аллелей.
\end{itemize}

\begin{table}[h]
\begin{center}
  \begin{tabular}{| p{2cm} | p{2cm}| p{9cm}|}
  \hline
     Аллель	& Значение & Аннотация \\ \hline
     1 & 14 & [GGTA]3 [GACA]6 [GATA]2 [GATT]3 \\ \hline
     2 & 17 & [GGTA]3 [GACA]8 [GATA]3 [GATT]3 \\ \hline
   \end{tabular}
  \end{center}
  \caption[c]{Пример аннотации локуса D5S2800}
  \label{table:loc_ann_ex}
\end{table}

\section{Подготовка данных}

В контексте работы над данной магистерской диссертацией был получен доступ к данным 48 образцов.
Каждый образец представляет собой результат так называемого секвенирования спаренных концов
(англ. paired-end sequencing), где помимо обычных прочтений в 'прямом' направлении,
присутствуют также прочтения в обратном направлении. Данная процедура позволяет повысить качество
выравнивания (англ. sequence alignment) на референсный геном. Таким образом для каждого
образца было получено 2 файла в формате $.fastq$.

Для извлечения прочтений соответствующих локусов, необходимо было выровнять все прочтения на референсный геном.
В открытом доступе находятся инструменты, позволяющие произвести данную операцию.
В качестве референса был использован геном \textbf{grch38}, для выравнивания наиболее
простым в использовании показался инструмент \textbf{bowtie2} (с интерфейсом командной строки).
В результате для каждого образца был получен результат выравнивания в формате $.sam$.

Далее, для работы с файлами в формате $.bam$/$.sam$ был использован инструмент \textbf{samtools}.
Для каждого образца после выравнивания были выполнены следующие операции:
\begin{itemize}
\item Перевод в формат $.bam$
\item Сортировка каждого $.bam$ файла
\item Построение индекса для каждого $.bam$ файла
\end{itemize}

На основе построенного индекса, для каждого образца появилась возможность выделять подмножество
прочтений относящихся к заданному локусу. В целях автоматизации процесса, была реализована
вспомогательная утилита на основе инструментов \textbf{bowtie2} и \textbf{samtools} для
обработки исходных секвенированных данных и выделения из них только интересующих нас локусов.
В результате обработки данных было установлено, что среди всех интересующих нас локусов
(X-хромосомы и аутосомных), для которых возникла необходимость в автоматизации процесса аннотации
STR-маркеров, были найдены прочтения только для 21 локуса:
\begin{itemize}
\item Локусы аутосом: D12ATA63, D14S1434, D1S1677, D2S1776, D3S4529, D5S2800, D6S474, D9S2157
\item Локусы X-хромосомы: DXS10075, DXS10079, DXS10101, DXS10134, DXS10146, DXS10147, DXS10148, DXS7133, DXS7424, DXS8377, DXS9895, GATA172D05
\end{itemize}

\section{Поиск шаблонов STR-маркеров в прочтениях}

На основании анализа имеющихся паттернов шаблонов было установлено, что общая схема
любого шаблона может быть описана последовательностью, где каждый элемент принадлежит одному из следующих классов:
\begin{itemize}
\item Тандемный повтор, аннотация имеет вид $[pattern]n$.
В случае, когда для некоторого локуса уже известен паттерн, необходимо лишь найти число повторений.
Можно считать, что в большинстве случаев длина паттерна составляет от 3 до 6 нуклеотидов.

\item Разделитель -- некоторая определенная последовательность нуклеотидов.

\item Подстановочный элемент -- последовательность любых символов определенной длины,
аннотация имеет вид $Nk$, где $k$ -- количество символов.
\end{itemize}

\begin{Example}
Референсный шаблон для локуса D12S391 имеет вид: $[AGAT]_{n} \, GAT \, [AGAT]_{n} \, [AGAC]_{n} \, AGAT$.
Можно заметить, что шаблон описан следующей последовательностью базовых классов:
тандемный повтор, разделитель, тандемный повтор, тандемный повтор, разделитель.
Пусть, было найдено вхождение шаблона выше в некоторое прочтение.
Аннотация будет представлять собой шаблон, но вместо символов $n$ будут подставлены уже
конкретные значения, характеризующие мощность каждого тандемного повтора. Например:
$[AGAT]_{12} \, GAT \, [AGAT]_{5} \, [AGAC]_{3} \, AGAT$.
Общая числовая характеристика данного аллеля будет равна сумме количеств всех вхождений, т.е.
значение данного аллеля будет равно 12 + 1 + 5 + 3 + 1 = 22.
В дальнейшем данную числовую характеристику совпадения будем называть мощностью.
\end{Example}

\begin{Remark}
Разделители обычно принимают за одно вхождение. Однако, если в записи шаблона разделитель
состоит из строчных букв, то при итоговом подсчете значения аллеля такой разделитель игнорируется.
\end{Remark}

\begin{Remark}
Вообще говоря для некоторого паттерна может существовать несколько вариантов вхождения его в некоторое
прочтение. Так, среди всех возможных вхождений будем отдавать предпочтение только имеющим
наибольшее численное значение.
\end{Remark}

Анализ описанной структуры шаблонов позволяет использовать регулярные выражения для поиска в прочтениях.
Так, среди всех найденных совпадений, удовлетворяющих заданному шаблону, необходимо найти
совпадение наибольшей мощности.

\section{Метод поиска возможных паттернов STR-маркеров}

Как уже было отмечено, иногда возникают ситуации когда сведения о референсном паттерне для
некоторого локуса отсутствуют. Одним из базовых подходов для решения данной проблемы является
генерация возможных шаблонов-кандидатов, а также последующая оценка кандидатов с целью выделения
достоверных шаблонов.

Для некоторого локуса, оценка качества шаблонов может осуществляться на следующих трех уровнях:
\begin{itemize}
\item Уровень одного прочтения. Простейшей метрикой для шаблона в данном случае может служить
мощность совпадения. Однако, поиск шаблона с наибольшей мощность может не дать в конечном счете
шаблон, который будет встречаться во всех образцах, исключая таким образов возможность аннотации.

\item Уровень одного образца. Как было отмечено выше, для каждого образца имеется
набор прочтений соответствующих заданному локусу. Частота обнаружения совпадений
во всех прочтениях для некоторого шаблона может служить более надежным индикатором
достоверности.

\item Уровень локуса. Для некоторого шаблона, на имеющемся наборе образцов и всех прочтений
для них, можно оценить встречаемость шаблона во всех образцах - в общем случае достоверными шаблонами можно считать
те, что встретились во всех образцах. Также, для каждого образца можно выделить совпадение наибольшей мощности,
просуммировав эту величину по всем образцам.
\end{itemize}

Имея некоторую интуицию касательно оценки возможных шаблонов кандидатов, можно рассмотреть
следующий итеративный метод генерации кандидатов для некоторого прочтения. Таким образом, после генерации
кандидатов для всех прочтений всех образцов, а также после удаления дубликатов среди полученного
множества кандидатов, используя предложенные метрики в результате мы получим некоторый
ограниченный список наиболее достоверных кандидатов. После экспертной оценки данного списка
можно принять выбранный шаблон за референсный.

В основе метода будут лежать следующие логические блоки:
\begin{itemize}
\item Нахождение шаблонов нулевого уровня. Шаблонами нулевого будем называть простые тандемные
повторы с длиной паттерна от 3 до 6, которые встречаются в прочтении с мощностью не менее 3.
Реализация данного блока состоит из двух шагов: нахождение всех мотивов в прочтении, а также
проверка найденных мотивов в качестве паттернов для тандемных повторов.

\item Проверка множества паттернов. Под проверкой будем подразумевать простую фильтрацию паттернов,
для которых не было найдено вхождений в заданное прочтение. Если вхождение нашлось,
то мы можем запомнить вхождение наибольшей мощности, а также его позицию в прочтении.

\item Расширение множества паттернов. Для каждого паттерна можно попробовать дополнить его
путем расширения либо слева, либо справа. Под расширением будем понимать приписывание к паттерну с соответствующей
стороны либо паттерна для тандемного повтора, либо тандемного повтора и разделителя.
Ввиду того, что для каждого шаблона мы знаем позицию вхождения в прочтение,
мы также владеем информацией об окрестности каждого вхождения, таким образом перебирая
длину паттерна тандемного повтора (а также разделителя), мы однозначно определяем сам
паттерн повтора.
\end{itemize}

\Algorithm Генерация возможных шаблонов STR-маркеров для одного прочтения
\begin{enumerate}
  \item Генерация шаблонов нулевого уровня
  \item Для шаблонов текущего уровня
  \begin{itemize}
    \item Проверяем множество шаблонов текущего уровня, удаляем шаблоны без вхождений.
    \item Если множество шаблонов на текущем уровне пустое -- переходим к пункту 3.
    \item Для каждого проверенного шаблона пробуем расширить его
    \item Производим дедубликацию сгенерированного множества шаблонов, относим их следующему уровню,
     переходим на следующий уровень и повторяем шаг 2
  \end{itemize}
  \item Результатом работы алгоритма являются все проверенные шаблоны, имеющие вхождения в прочтение
\end{enumerate}

Алгоритм показал хорошие результаты на имеющихся данных,
для локусов с подтвержденными референсными шаблонами были получены идентичные, либо схожие шаблоны
STR-повторов.

\section{Аннотация STR-маркеров}

Для заданного локуса, а также в случае наличия референсного шаблона, задача аннотации представляет
собой получения строкового представления вхождения шаблона в каждый из аллелей. Возможные
сложности могут быть связаны с наличием ошибок в прочтениях, а также тем обстоятельством, что прочтения
для гомологических хромосом не разделены, таким образом появляется также задача разделения аллелей
и аннотаций для них.

\Algorithm Метод аннотации STR-маркеров заданного локуса
\begin{enumerate}
  \item Подсчет частоты встречаемости каждого прочтения.
  В данном случае не производится никакой работы
  по обнаружению и отделению фланкирующих регионов в соответствующих прочтениях. Происходит отсев
  прочтений согласно некоторым определенным заранее правилам, например, использование порогового значения.

  \item Для каждого прочтения производится поиск паттерна (также производится учет комплиментарных прочтений),
  сохраняется аннотация для полученной мощности совпадения.

  \item Определение гетерозиготности рассматриваемого образца.
  Если среди множества полученных значений аллелей явно выделяются 2, то заключаем,
  то организм гетерозиготен для данного STR-маркеру,
  иначе -- гомозиготен.

  \item В соответствии с установленными на предыдущем шаге свойствами образца,
  необходимо вернуть в качестве результата аннотации для необходимого числа аллелей.
\end{enumerate}

Для тестирования описанных методов были разработаны и реализованы следующие инструменты:
\begin{enumerate}
  \item Система для аннотации STR-маркеров.
  Необходимо наличие референсной базы с указанием шаблонов для каждого из рассматриваемых локусов.

  \item Веб-приложение для анализа прочтений для некоторого локуса.

  Пользователь может загрузить некоторый файл с прочтениями в формате $.fastq$.
  Для 5 наиболее часто встречающихся прочтений пользователь получает возможность
  для исследования возможных шаблонов STR-маркеров с использованием регулярных выражений.
  Также, пользователь получает возможность для заданного множества прочтений получить
  множество наиболее вероятных шаблонов для рассматриваемого STR-маркера.

\end{enumerate}

\section{Выводы}

В данной главе были рассмотрены методы работы с STR-маркерами. Предложенные методы поиска шаблонов
STR-маркеров и их аннотации были реализованы и протестированы на реальных геномных данных.
Были разработаны и реализованы программные компоненты и инструменты на основе предложенных методов.
