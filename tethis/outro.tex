\chapter*{Заключение}
\addcontentsline{toc}{chapter}{Заключение}


В рамках магистерской диссертации были достигнуты следующие результаты:
\begin{enumerate}
    \item В первой главе рассмотрена общая схема ДНК-идентификации.
    \item Подробно изучены общие подходы к решению задачи обработки последовательностей нуклеотидов, получен практический опыт работы с основными инструментами.
    \item Сформулирована общая схема технологии ДНК-идентификации, выделены подзадачи для реализации.
    \item Во второй главе описаны разработанные методы для поиска шаблонов STR-маркеров, а также их аннотации.
    \item Разработанные методы для работы с STR-маркерами реализованы в виды веб-приложения.
    \item В третьей главе описаны результаты проведенного анализа генотипов, предварительно проведена очистка размеченных данных.
    \item Установлено качество классификации генотипов с целью определения этногеографической принадлежности.
    \item В четвертой главе описывается задача определения географической принадлежности генотипов.
    \item Предложены базовые метрики оценки качества решения, разработаны методы решения.
\end{enumerate}
\clearpage

\addcontentsline{toc}{chapter}{Список использованной литературы}
\renewcommand{\bibname}{Список использованной литературы}
\begin{thebibliography}{9}

\bibitem{rt}
Ахо А.В. Структуры данных и алгоритмы
/ А. В. Ахо, Д. Э. Хопкрофт, Д. Д. Ульман. : Учеб. пособие/ пер. с англ. М. : Вильямс, 2000. – 384 с.
\bibitem{rt}
Кормен, Т. Алгоритмы : построение и анализ
/ Т. Кормен, Ч. Лейверсон, Р. Ривест, К. Штайн: Вильямс, 2005. 1296 с.
\bibitem{rt}
Christopher Bishop. Pattern Recognition and Machine Learning
/ Springer, 2006
\bibitem{rt}
Phillip Compeau, Pavel Pevzner. Bioinformatics algorithms: an active learning approach
/ La Jolla, CA : Active Learning Publishers, 2014. 362 с.
\bibitem{rt}
Benjamin M. Peter., 2016: Admixture, Population Structure, and F-Statistics
/ Genetics, Vol. 202, 1485–1501
\bibitem{rt}
Nick Patterson et al., 2012: Ancient Admixture in Human History
/ Genetics, Vol. 192, 1065–1093
\bibitem{rt}
Ding et al., 2011: Comparison of measures of marker informativeness for ancestry and admixture mapping.
/  BMC Genomics 12:622.
\bibitem{rt}
Pickrell JK, Pritchard JK, 2012: Inference of Population Splits and Mixtures from Genome-Wide Allele Frequency Data. / / / PLoS Genet 8(11)
\bibitem{rt}
David H. Alexander, John Novembre and Kenneth Lange, 2009: Fast model-based estimation of ancestry in unrelated individuals
/ Genome Res. 19: 1655-1664
\bibitem{rt}
Rosenberg NA, Pritchard JK, Weber JL, Cann HM, Kidd KK, Zhivotovsky LA, Feldman MW, 2002: Genetic structure of
human populations.
/ Science 298: 2381–2385
\bibitem{rt}
Rosenberg et al., 2003: Marker Informativeness for Inference of Ancestry
/ Am. J. Hum. Genet. 73:1402–1422, 2003
\bibitem{rt}
Jonathan K. Pritchard, et al., 2012: Inference of Population Structure Using Multilocus Genotype Data
/ Genetics 155: 945–959
\bibitem{rt}
Bioinformatics Specialization (MOOC) /
Режим доступа: https://www.coursera.org/specializations/bioinformatics – Дата доступа: 01.11.2018.
\bibitem{rt}
Y-Chromosome STR Haplotype Reference Database /
Режим доступа: http://yhrd.org – Дата доступа: 25.12.2018.
\bibitem{rt}
Principal component analysis /
Режим доступа: http://en.wikipedia.org/wiki/Principal component analysis – Дата доступа: 14.01.2019.
\bibitem{rt}
t-distributed stochastic neighbor embedding /
Режим доступа: http://en.wikipedia.org/wiki/T-distributed stochastic neighbor embedding – Дата доступа: 19.01.2019.
\bibitem{rt}
Population genetics and genomics in R /
Режим доступа: https://grunwaldlab.github.io/Population Genetics in R/Pop Structure.html – Дата доступа: 23.01.2019.
\bibitem{rt}
mmod: Modern Measures of Population Differentiation /
Режим доступа: https://cran.r-project.org/web/packages/mmod/index.html – Дата доступа: 26.01.2019.
\bibitem{rt}
Neighbor joining /
Режим доступа: http://en.wikipedia.org/wiki/Neighbor joining – Дата доступа: 28.01.2019.
\bibitem{rt}
STRUCTURE Software /
Режим доступа: https://web.stanford.edu/group/pritchardlab/structure.html – Дата доступа: 04.02.2019.
\end{thebibliography}

\chapter*{ПРИЛОЖЕНИЕ А}
\addcontentsline{toc}{chapter}{ПРИЛОЖЕНИЕ А}

\textbf{Ссылка на реализацию разработанных инструментов}

\href{https://github.com/dzemidkada/bsu-genomics}{https://github.com/dzemidkada/bsu-genomics}
